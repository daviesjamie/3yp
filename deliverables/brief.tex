\documentclass[a4paper,12pt]{article}

\usepackage{hyperref}
\usepackage{cite}
\usepackage[margin=1in]{geometry}

\title{
    \Large{Part III Individual Project Brief}\\[0.5cm]
    \huge{Enhanced Content Analysis and Information Retrieval Using Twitter Hashtags}
}
\date{\vspace{-5ex}}
\author{
    \emph{Author:}\\
    Jamie Davies\\
    \href{mailto:jagd1g11@ecs.soton.ac.uk}{\nolinkurl{jagd1g11@ecs.soton.ac.uk}}
    \and
    \emph{Supervisor:}\\
    Dr. Nicholas Gibbins\\
    \href{mailto:nmg@ecs.soton.ac.uk}{\nolinkurl{nmg@ecs.soton.ac.uk}}
}

\begin{document}
\pagenumbering{gobble}

\maketitle
\vspace*{0.5cm}

Microblogging services such as Twitter\footnote{\url{http://www.twitter.com}} have become very popular in recent years. One of the key characterists of such services is the use of `hashtags' --- topical/categorical annotations provided by the authors of the posts~(tweets) themselves. These hashtags have been proven to be useful and information-rich in previous microblogging research projects; from using them to crowd-source real-time event detection~\cite{Sakaki:2010}, to using them to train sentiment classifiers~\cite{Davidov:2010}. However due to their nature, there is a vast and diverse collection of hashtags for users to choose from, resulting in the information provided by hashtags not being as accurate or as complete as it could be.

\bibliographystyle{abbrv}
{\footnotesize \bibliography{bibliography}}
\end{document}
