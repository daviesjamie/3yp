\documentclass[12pt,a4paper]{article}

\usepackage{cite}
\usepackage{parskip}
\usepackage[hidelinks]{hyperref}
\usepackage[margin=1.5in]{geometry}
\usepackage{graphicx}

\begin{document}

\begin{titlepage}
\center
\vspace*{3cm}

{\Large
    Electronics and Computer Science\\
    Faculty of Physical Sciences and Engineering\\
    University of Southampton\\[1cm]
}

Jamie Davies\\
%\href{mailto:jagd1g11@ecs.soton.ac.uk}{\nolinkurl{jagd1g11@ecs.soton.ac.uk}}\\[3cm]
\today\\[1cm]

{\large
    Enhanced Content Analysis and Information\\
    Retrieval Using Twitter Hashtags\\[1cm]
}

A project report submitted for the award of\\
MEng Computer Science\\[1cm]

\emph{Supervisor:}\\
Dr. Nick Gibbins\\[0.5cm]
%\href{mailto:nmg@ecs.soton.ac.uk}{\nolinkurl{nmg@ecs.soton.ac.uk}}\\[0.5cm]

\emph{Examiner:}\\
Dr. Otto Octavius\\
%\href{mailto:ogo@ecs.soton.ac.uk}{\nolinkurl{ogo@ecs.soton.ac.uk}}\\


\vfill
\end{titlepage}

\setcounter{secnumdepth}{0}
\section{Abstract}
One of the key characteristics of Twitter and other microblogging platforms is the use of `hashtags' --- topical/categorical annotations provided by the authors of the posts (tweets) themselves. As hashtags are simply any combination of letters and digits preceded by a hash (#) symbol, users are completely free to use whatever hashtags they like when publishing their tweets. Whilst this flexible tagging system gives users the freedom to express themselves accurately, it unfortunately also results in the data from the hashtags being fragmented and inaccurate due to the vast and diverse possibilities of hashtags for the user to choose from.

If users are presented with a choice of hashtags relevant to the content of the tweet they are writing, they are more likely to publish tweets with accurate tag data. This opens an opportunity for an intelligent hashtag suggestion tool to substantially raise the information gain from hashtags.

However, whilst such a system could 
\pagebreak

\tableofcontents
\pagebreak

\setcounter{secnumdepth}{1}
\section{Project Goals}
12pt text for main body.
\pagebreak

\section{Background and report of literature search}
12pt text for main body.
\pagebreak

\section{Report on Technical Progress}
12pt text for main body.
\pagebreak

\section{Plan of remaining work}
12pt text for main body.
\pagebreak

\setcounter{secnumdepth}{0}
\section{References}

\end{document}

